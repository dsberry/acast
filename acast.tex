%% This is file `elsarticle-template-2-harv.tex',
%%
%% Copyright 2009 Elsevier Ltd
%%
%% This file is part of the 'Elsarticle Bundle'.
%% ---------------------------------------------
%%
%% It may be distributed under the conditions of the LaTeX Project Public
%% License, either version 1.2 of this license or (at your option) any
%% later version.  The latest version of this license is in
%%    http://www.latex-project.org/lppl.txt
%% and version 1.2 or later is part of all distributions of LaTeX
%% version 1999/12/01 or later.
%%
%% The list of all files belonging to the 'Elsarticle Bundle' is
%% given in the file `manifest.txt'.
%%
%% Template article for Elsevier's document class `elsarticle'
%% with harvard style bibliographic references
%%
%% $Id: elsarticle-template-2-harv.tex 155 2009-10-08 05:35:05Z rishi $
%% $URL: http://lenova.river-valley.com/svn/elsbst/trunk/elsarticle-template-2-harv.tex $
%%

%%\documentclass[preprint,authoryear,12pt]{elsarticle}

%% Use the option review to obtain double line spacing
%% \documentclass[authoryear,preprint,review,12pt]{elsarticle}

%% Use the options 1p,twocolumn; 3p; 3p,twocolumn; 5p; or 5p,twocolumn
%% for a journal layout:

%% Astronomy & Computing uses 5p
%% \documentclass[final,authoryear,5p,times]{elsarticle}
\documentclass[final,authoryear,5p,times,twocolumn]{elsarticle}

%% if you use PostScript figures in your article
%% use the graphics package for simple commands
%% \usepackage{graphics}
%% or use the graphicx package for more complicated commands
\usepackage{graphicx}
%% or use the epsfig package if you prefer to use the old commands
%% \usepackage{epsfig}

%% The amssymb package provides various useful mathematical symbols
\usepackage{amssymb}
%% The amsthm package provides extended theorem environments
%% \usepackage{amsthm}

\usepackage[pdftex,pdfpagemode={UseOutlines},bookmarks,bookmarksopen,colorlinks,linkcolor={blue},citecolor={green},urlcolor={red}]{hyperref}
\usepackage{hypernat}

%% The lineno packages adds line numbers. Start line numbering with
%% \begin{linenumbers}, end it with \end{linenumbers}. Or switch it on
%% for the whole article with \linenumbers after \end{frontmatter}.
%% \usepackage{lineno}

%% natbib.sty is loaded by default. However, natbib options can be
%% provided with \biboptions{...} command. Following options are
%% valid:

%%   round  -  round parentheses are used (default)
%%   square -  square brackets are used   [option]
%%   curly  -  curly braces are used      {option}
%%   angle  -  angle brackets are used    <option>
%%   semicolon  -  multiple citations separated by semi-colon (default)
%%   colon  - same as semicolon, an earlier confusion
%%   comma  -  separated by comma
%%   authoryear - selects author-year citations (default)
%%   numbers-  selects numerical citations
%%   super  -  numerical citations as superscripts
%%   sort   -  sorts multiple citations according to order in ref. list
%%   sort&compress   -  like sort, but also compresses numerical citations
%%   compress - compresses without sorting
%%   longnamesfirst  -  makes first citation full author list
%%
%% \biboptions{longnamesfirst,comma}

% \biboptions{}

\journal{Astronomy \& Computing}

%% Make single quotes look right in verbatim mode
\usepackage{upquote}

\begin{document}

\begin{frontmatter}

%% Title, authors and addresses

%% use the tnoteref command within \title for footnotes;
%% use the tnotetext command for the associated footnote;
%% use the fnref command within \author or \address for footnotes;
%% use the fntext command for the associated footnote;
%% use the corref command within \author for corresponding author footnotes;
%% use the cortext command for the associated footnote;
%% use the ead command for the email address,
%% and the form \ead[url] for the home page:
%%
%% \title{Title\tnoteref{label1}}
%% \tnotetext[label1]{}
%% \author{Name\corref{cor1}\fnref{label2}}
%% \ead{email address}
%% \ead[url]{home page}
%% \fntext[label2]{}
%% \cortext[cor1]{}
%% \address{Address\fnref{label3}}
%% \fntext[label3]{}

\title{AST: A library for modelling and manipulating coordinate systems}

%% use optional labels to link authors explicitly to addresses:
%% \author[label1,label2]{<author name>}
%% \address[label1]{<address>}
%% \address[label2]{<address>}

\author[jac]{David S.\ Berry\corref{cor1}}
\ead{d.berry@jach.hawaii.edu}
\author[ral]{Rodney F.\ Warren-Smith}
\author[cornell,jac]{Tim Jenness}

\cortext[cor1]{Corresponding author}

\address[jac]{Joint Astronomy Centre, 660 N.\ A`oh\=ok\=u Place, Hilo, HI
  96720, USA}
\address[ral]{RAL Space, STFC Rutherford Appleton Laboratory, Harwell Oxford, Didcot, Oxfordshire OX11 0QX, UK}
\address[cornell]{Department of Astronomy, Cornell University, Ithaca,
  NY 14853, USA}

\begin{abstract}
This paper describes the Starlink AST library.
\end{abstract}

\begin{keyword}
%% keywords here, in the form: keyword \sep keyword

%% MSC codes here, in the form: \MSC code \sep code
%% or \MSC[2008] code \sep code (2000 is the default)

WCS \sep data models \sep Starlink

\end{keyword}

\end{frontmatter}

% \linenumbers

%% Journal abbreviations
\newcommand{\mnras}{Mon Not R Astron Soc}
\newcommand{\aap}{Astron Astrophys}
\newcommand{\aaps}{Astron Astrophys Supp}
\newcommand{\pasp}{Pub Astron Soc Pacific}
\newcommand{\apj}{Astrophys J}
\newcommand{\apjs}{Astrophys J Supp}
\newcommand{\qjras}{Quart J R Astron Soc}
\newcommand{\an}{Astron.\ Nach.}
\newcommand{\ijimw}{Int.\ J.\ Infrared \& Millimeter Waves}
\newcommand{\procspie}{Proc.\ SPIE}
\newcommand{\aspconf}{ASP Conf. Ser.}

%% ASCL
\newcommand{\ascl}[1]{\href{http://www.ascl.net/#1}{ascl:#1}}

%% main text
\section{Introduction}
\label{sec:intro}
The Starlink AST library \citep{SUN211} ...

\section{Motivation}

\section{What is AST?}

AST is a library that provides a generalised scheme for modelling and
manipulating inter-related coordinate systems. Obvious use in
astronomy is WCS, but coordinate systems are used in other places too
(e.g. graphics).  AST includes facilities for importing and (where
possible) exporting a wide variety of FITS-WCS headers, including
various popular distortion schemes. AST was first released in 1997
\cite[][included in ``Twenty Years of ADASS'']{1998ASPC..145...41W}.
So 16 years of use within application code including Starlink, DS9 and
XIMAGE. Interest in more flexible schemes for representing
inter-related coordinate systems has increased recently (cf astropy
and alternatives to FITS), so it is a good time to review the lessons
learned from AST.

Flexible handling of coordinate systems is important for describing
raw data (distortions etc), but is also very important for allowing
creative use of coordinate transformations during data analysis. Also
important for handling of slices, and dimensionality changes.

\section{Key Classes within AST}

\subsection{Key principles}

Distinguish clearly between the purposes of Mappings and Frames.

Think in terms of Frames rather than axes. FITS-WCS tries to maintain
some sort of 1-to-1 correspondence between pixel axes and WCS axes,
and specifies the properties of each axis individually. But in general
there is no such correspondence (think of the case of a square image
centred on the north pole - which pixel axis corresponds to RA and
which to Dec?).

\subsection{Frames and Domains}

\subsubsection{What is a Domain?}

A physical space such as ``time'', ``the sky'', ``the EM spectrum'', ``the
focal plane'', ``a pixel array''. Points within such a space can in
general be described using any one of several coordinate systems
(e.g. the sky can be described using ICRS, Galactic, etc, the EM
spectrum can be described using freq, wavelength, etc). The ``Frame''
class represents a domain, and should encapsulate all the metadata
needed to transform a position between any pair of coordinate systems
that can be used to describe positions within its domain (in
retrospect, we should have named the class ``Domain'' rather than
``Frame''). Thus things like ``pixel size'' should \emph{not} be stored in a
Frame (Arnold), since it is a property of the relationship between two
Frames (a.k.a the Mapping) rather than of the Frame itself.

Basic Frame properties and methods:

\begin{itemize}
\item Sky, Time and Spec Frames
\item Compound Frames
\item Converting between Frames
\end{itemize}

\subsection{Mappings}

Mapping methods

Atomic Mappings

Compound Mappings (inc. SwitchMaps)

\subsubsection{Simplification}

There are a wide range of possible transformations that could
potentially be applied to a data set data during analysis. These
including simple things such as rotation, scaling, shear, etc., but
could in principle include more complex transformations such as
re-projection, dis-contiguous ``patchwork'' transformations, or even
transformation using a general algebraic expression.  A coordinate
handling system should make it possible for a user to apply an
arbitrary set of such transformations in series to a data set, without
losing track of the coordinates of each data point. With a
prescriptive scheme such as FITS-WCS this would require each
transformation to locate the appropriate component of the FITS-WCS
pixel to world coordinate mapping, and modify the corresponding
headers in a suitable way. This is often a difficult, if not
impossible, task. Within AST, the chaining of transformations is
accomplished simply by creating a Mapping that describes each new
transformation and concatenating it with the existing pixel to world
coordinate mapping.

However, by itself this can lead to the mapping becoming increasingly
complex as transformations are stacked on top of each other. This is
bad because it leads to
\begin{enumerate}
\item slower evaluation of the total transformation,
 \item less accurate evaluation of the total
transformation, and
\item more room being needed to store the total
transformation.
\end{enumerate}
To avoid this, the Mapping class provides a
``Simplify'' method that takes a potentially complex Mapping and
simplifies it as far as possible. Doing such simplification in a
general and effective manner is one of the most difficult challenges
faced by the AST model, but experience has shown that the current
scheme implemented in AST handles most cases sufficiently well. The
steps involved in simplification depend on the nature of the component
Mappings in the total Mapping. Each class of Mapping provides its own
rules that indicate when and how it can be simplified, or combined
with an adjacent Mapping in the chain. To illustrate the principle,
some of the simplest examples include,
\begin{enumerate}
 \item any Mapping can be combined with its own inverse to create a
   UnitMap,
\item UnitMaps can be removed entirely,
\item adjacent MatrixMaps can be combined using matrix
multiplication to create a single MatrixMap,
\item adjacent ShiftMaps can be combined to form a single ShiftMap.
\end{enumerate}

The whole simplification process is managed by the Simplify method of
the CmpMap class. It expands the compound Mapping into a list of
atomic Mappings to be applied in series or parallel, and then for each
Mapping in the list, invokes that Mapping’s protected ``MapMerge''
method. This method is supplied with the entire list of atomic
Mappings, and determines if the nominated Mapping can be merged with
any of its neighbours. If so, a new list of Mappings is returned
containing the merged Mapping in place of the original mappings. Once
all atomic Mappings in the CmpMap have been checked in this way, the
same process is repeated again from the beginning in case any of the
changes that have been made to the list allow further simplifications
to be performed. This process is repeated until no further
simplifications occur.

\subsection{FrameSets}

\subsubsection{Base and Current Frames}

The ``base'' Frame represents the coordinate system in which axis values
are initially generated or obtained. For instance a cursor application
generates positions initially in graphics screen coordinates, and a
centroiding application generates them in pixel coordinates.  The
``current'' Frame represents the coordinate system in which positions
are required by subsequent user (code or human). This will often be
some form of WCS, such as (RA,Dec), but could potentially be any of
the Frames available in the FrameSet.

\subsubsection{Integrity restoration}

Changing a Frame to represent a different coordinate system should
cause the Mappings that connect it to the other Frames in the FrameSet
to be updated.

\subsubsection{Using a FrameSet to calibrate a data cube or table}

The base Frame represents grid coords, and the current Frame
represents WCS coords.  Explain about the usefulness of PIXEL versus
GRID coordinates.

\subsection{Regions}

These are the objects that contain actual axis values. They
encapsulate a Frame and set of positions defined within the
Frame. They can be transformed automatically into other Frames by
either changing the properties of the Frame or by using a Mapping to
map them into a new specified Frame.

The simplest subclass of Region is the PointList that represents one
or more positions within the Frame.  In practice, PointLists are
rarely used. This is because a typical application is usually
interested in positions that it has determined itself in some way, and
so can process them directly using Mappings without the need to create
an intermediate container object such as a PointList. Applications
that for instance displays the world coordinates of the cursor, or the
centroid position of an object, will usually obtain the axis values in
some ``base'' system such as graphics screen coordinates or pixel
coordinates, and will then transform them directly into the coordinate
system of interest using a suitable Mapping that it has obtained
previously. There is no need to first wrap the axis values in a
PointList, transform the PointList to a new coordinate system, and
then extract the coordinates from the transformed PointList. Doing so
can potentially introduces large overheads that can increase the
overall time taken to do the transformation.

Other subclasses include, Circle, Box, Ellipse, etc. Compound Regions
can be formed to represent the union or intersection of two
Regions. The Prism class allows Regions to be extruded into other
dimensions.

Regions can behave as either Mappings or Frames. When Used as a
Mapping, positions within the Region are left unchanged, and positions
outside the Region are set to a ``bad'' value.

Regions can contain a FrameSet in place of a Frame. In this case, the
shape and extent of the Region are defined within the base Frame of
the FrameSet, and the Region notionally represents a ``view'' of this
shape transformed into the current Frame of the FrameSet. For
instance, a ``Box'' could be created to represent the rectangular extent
of an image in pixel coordinates. Such a Box will contain a Frame
describing pixel coordinates, the pixel coordinates at the centre of
the image, and the extent of each dimension of the image (in
pixels). The Frame in this Box could be replaced by a FrameSet in
which the base Frame is again pixel coordinates, but the current Frame
is (RA,Dec). Internally, the Box still represents a rectangular area
in pixel coordinates, but is now viewed externally as the
corresponding curvilinear area in the (RA,Dec) Frame. When the Box is
used as a Mapping, it will first transform each position from (RA,Dec)
into pixel coordinates, and then test these transformed positions to
see if they are within the rectangular area of pixel coordinates
described by the Box.

\section{Fields of Application}



\subsection{Generalised Plotting}

\section{Things we would do differently now}

\begin{enumerate}

\item Be clearer about the distinction between a coordinate system and
  a domain. Coordinate systems are mathematical abstractions of
  various types (Cartesian, polar, etc). You use a coordinate system
  to describe a position within a physical space (domain). In this
  sense a domain can encapsulate several alternative coordinate
  systems, any of which can be used to describe positions in the
  domain.

\item Have a specific sub-class of Frame to describe a pixel array ( a
  PixelFrame), allowing System=GRID or System=PIXEL.

\item Use degrees instead of radians for sky axes.

\item The restriction that Mappings can only transform doubles may be
  a problem for time axes (but this is an issue with the
  implementation rather than the model).

\item Make it more modular. Some sort of facility for optional
  extensions or plug-ins, so that the thing does not become such a
  huge monolithic lump.

\end{enumerate}

\section{The AST library}

The AST library has been developed over a number of years
\citep{1998ASPC..145...41W,2000ASPC..216..506W,2001ASPC..238..129B,2004ASPC..314..412B,2008ASPC..394..635B,2010ASPC..434..213B,2012ASPC..461..825B}
  and is written in C with no dependencies. It includes code from
  WCSLIB \citep{2006ASPC..351..591C}, PAL \citep{2013ASPC..475..307J}
  and SOFA \citep{2011SchpJ...611404H} but does not depend on those
  libraries. There are language bindings for Fortran, Java, Perl and
  Python.

\section{Acknowledgements}

The Starlink software is currently maintained by the Joint Astronomy
Centre, Hawaii with support from the UK Science and Technology
Facilities Council.


%% The Appendices part is started with the command \appendix;
%% appendix sections are then done as normal sections
%% \appendix

%% \section{}
%% \label{}

%% References
%%
%% Following citation commands can be used in the body text:
%%
%%  \citet{key}  ==>>  Jones et al. (1990)
%%  \citep{key}  ==>>  (Jones et al., 1990)
%%
%% Multiple citations as normal:
%% \citep{key1,key2}         ==>> (Jones et al., 1990; Smith, 1989)
%%                            or  (Jones et al., 1990, 1991)
%%                            or  (Jones et al., 1990a,b)
%% \cite{key} is the equivalent of \citet{key} in author-year mode
%%
%% Full author lists may be forced with \citet* or \citep*, e.g.
%%   \citep*{key}            ==>> (Jones, Baker, and Williams, 1990)
%%
%% Optional notes as:
%%   \citep[chap. 2]{key}    ==>> (Jones et al., 1990, chap. 2)
%%   \citep[e.g.,][]{key}    ==>> (e.g., Jones et al., 1990)
%%   \citep[see][pg. 34]{key}==>> (see Jones et al., 1990, pg. 34)
%%  (Note: in standard LaTeX, only one note is allowed, after the ref.
%%   Here, one note is like the standard, two make pre- and post-notes.)
%%
%%   \citealt{key}          ==>> Jones et al. 1990
%%   \citealt*{key}         ==>> Jones, Baker, and Williams 1990
%%   \citealp{key}          ==>> Jones et al., 1990
%%   \citealp*{key}         ==>> Jones, Baker, and Williams, 1990
%%
%% Additional citation possibilities
%%   \citeauthor{key}       ==>> Jones et al.
%%   \citeauthor*{key}      ==>> Jones, Baker, and Williams
%%   \citeyear{key}         ==>> 1990
%%   \citeyearpar{key}      ==>> (1990)
%%   \citetext{priv. comm.} ==>> (priv. comm.)
%%   \citenum{key}          ==>> 11 [non-superscripted]
%% Note: full author lists depends on whether the bib style supports them;
%%       if not, the abbreviated list is printed even when full requested.
%%
%% For names like della Robbia at the start of a sentence, use
%%   \Citet{dRob98}         ==>> Della Robbia (1998)
%%   \Citep{dRob98}         ==>> (Della Robbia, 1998)
%%   \Citeauthor{dRob98}    ==>> Della Robbia


%% References with bibTeX database:

\bibliographystyle{model2-names-astronomy}
\bibliography{acast}

%% Authors are advised to submit their bibtex database files. They are
%% requested to list a bibtex style file in the manuscript if they do
%% not want to use model2-names.bst.

%% References without bibTeX database:

% \begin{thebibliography}{00}

%% \bibitem must have one of the following forms:
%%   \bibitem[Jones et al.(1990)]{key}...
%%   \bibitem[Jones et al.(1990)Jones, Baker, and Williams]{key}...
%%   \bibitem[Jones et al., 1990]{key}...
%%   \bibitem[\protect\citeauthoryear{Jones, Baker, and Williams}{Jones
%%       et al.}{1990}]{key}...
%%   \bibitem[\protect\citeauthoryear{Jones et al.}{1990}]{key}...
%%   \bibitem[\protect\astroncite{Jones et al.}{1990}]{key}...
%%   \bibitem[\protect\citename{Jones et al., }1990]{key}...
%%   \harvarditem[Jones et al.]{Jones, Baker, and Williams}{1990}{key}...
%%

% \bibitem[ ()]{}

% \end{thebibliography}

\end{document}

%%
%% End of file `elsarticle-template-2-harv.tex'.
